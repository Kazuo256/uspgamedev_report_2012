\documentclass[12pt,onecolumn,a4paper]{article}
\usepackage[brazilian]{babel}
\usepackage[utf8]{inputenc}
\usepackage{hyperref}
%\usepackage{amsmath}


\begin{document}

% Title page.
\begin{titlepage}

    % Title info.
    \title{
        \bf
        \Huge  USPGameDev \\
        \large Grupo de Pesquisa e Desenvolvimento de Jogos da USP \\
        \LARGE Relatório
    }
    \author{Wilson Kazuo Mizutani}
    
    % Print title.
    \maketitle
    
    % No numbering on this page.
    \thispagestyle{empty}
    
\end{titlepage}

% Print table of contents.
\tableofcontents

% Page break.
\clearpage

% O Giuliano pediu para fazer o relatório em duas partes, uma com explicações e outra com
% resultados (usando imagens). A primeira ele estimou umas 5 páginas, e, para a segunda, 20.

%%% Parte I: Explicações. %%%

\section{\LARGE O que é o USPGameDev?}

    \subsection{Definição e objetivos}
        O USPGameDev é um grupo de pesquisa e desenvolvimento de jogos da USP. Ele tem como meta
        criar um ambiente no qual membros da comunidade USP interessados na criação de jogos
        eletrônicos possam por em prática o que aprenderam em seus cursos e desenvolver suas
        próprias ideias. Além disso, o grupo visa aprender sobre práticas de gerenciamento de
        equipes e projetos, sobre processos de elaboração de jogos e, também, adquirir familiaridade
        com as atuais ferramentas de desenvolvimento utilizadas na área.
    
    \subsection{Filosofia}
        O grupo procura adotar uma filosofia de {\it Software Livre} - permitindo que qualquer um
        tenha acesso a estudar, modificar e redistribuir aquilo que produzimos - e de {\it
        Portabilidade} - garantindo que aquilo que produzimos possa ser compatível com a maior parte
        possível dos sistemas operacionais e arquiteturas encontradas nos computadores modernos. É
        do interesse do grupo que o maior número possível de pessoas possam usufruir dos nossos
        resultados, sem restruições.
    
    \subsection{Atividades}
        A rotina usual do grupo consiste em reuniões presenciais, três vezes por semana, para
        discussão e trabalho.
    
    % Metodologia.

\section{\LARGE Histórico}
    % TODO

\section{\LARGE Membros}
    % TODO

\section{\LARGE Projetos}
    % TODO

%%% Parte II: Resultados. %%%

\section{\LARGE Resultados}

% End of document.
\end{document}

