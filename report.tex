\documentclass[12pt,onecolumn,a4paper]{article}
\usepackage[brazilian]{babel}
\usepackage[utf8]{inputenc}
\usepackage{hyperref}
%\usepackage{amsmath}


\begin{document}

% Title page.
\begin{titlepage}

    % Title info.
    \title{
        \bf
        \Huge  USPGameDev \\
        \large Grupo de Pesquisa e Desenvolvimento de Jogos da USP \\
        \LARGE Relatório
    }
    \author{Wilson Kazuo Mizutani}
    
    % Print title.
    \maketitle
    
    % No numbering on this page.
    \thispagestyle{empty}
    
\end{titlepage}


% Print table of contents.
\tableofcontents

% Page break.
\clearpage

% Make pages from now on have a header.
\pagestyle{myheadings}
\markright{USPGamedev - Relatório}

% O Giuliano pediu para fazer o relatório em duas partes, uma com explicações e outra com
% resultados (usando imagens). A primeira ele estimou umas 5 páginas, e, para a segunda, 20.

%%% Parte I: Explicações. %%%

\section{\LARGE O que é o USPGameDev?}

    \subsection{Definição e objetivos}
        O USPGameDev é um grupo de pesquisa e desenvolvimento de jogos da USP. Ele tem como meta
        criar um ambiente no qual membros da comunidade USP interessados na criação de jogos
        eletrônicos possam por em prática o que aprenderam em seus cursos e desenvolver suas
        próprias ideias. Além disso, o grupo visa aprender sobre práticas de gerenciamento de
        equipes e projetos, sobre processos de elaboração de jogos e, também, adquirir familiaridade
        com as atuais ferramentas de desenvolvimento utilizadas na área.
    
    \subsection{Filosofia}
        O grupo procura adotar uma filosofia de {\it Software Livre} - permitindo que qualquer um
        tenha acesso a estudar, modificar e redistribuir aquilo que produzimos - e de {\it
        Portabilidade} - garantindo que aquilo que produzimos possa ser compatível com a maior parte
        possível dos sistemas operacionais e arquiteturas encontradas nos computadores modernos. É
        do interesse do grupo que o maior número possível de pessoas possam usufruir dos nossos
        resultados, sem restruições.
    
    \subsection{Atividades}
        A rotina usual do grupo consiste em reuniões presenciais, três vezes por semana, para
        discussão e trabalho. Ou seja, é nessas reuniões que ocorre o desenvolvimento dos projetos
        nos quais o grupo está envolvido, assim como eventuais debates sobre metas futuras e suas
        prioridades. No entanto, além dos projetos centrais, há outras atividades nas quais o grupo
        participa.
        
        A primeira delas é a realização de palestras informativas para divulgar o trabalho do grupo
        e compartilhar experiências. Desde o seu primeiro lançamento, o grupo tem se esforçado para
        encontrar seu lugar na comunidade USP, realizando apresentações como essas inclusive no
        campus da EACH.
        
        Outra atividade é o oferecimento de cursos técnicos, também voltados ao público da USP. O
        conteúdo desses cursos abrange ferramentas de desenvolvimento de software como:

        \begin{itemize}
        
            \item{\bf CMake \footnotemark} - gerenciador de construção de projetos de software
                                             multiplataforma;
            \footnotetext{Site oficial: \url{www.cmake.org}}
            
            \item {\bf Lua \footnotemark} - linguagem de script brasileira muito usada por jogos de
                                            renome no mercado; e
            \footnotetext{Site oficial: \url{www.lua.org}}

            \item {\bf LÖVE \footnotemark} - uma plataforma para desenvolvimento de jogos em duas
                                             dimensões minimal e gratuita.
            \footnotetext{Site oficial: \url{www.love2d.org}}
            
        \end{itemize}
        
        Mas também inclui ferramentas de propósito mais geral como:
        
        \begin{itemize}
        
            \item {\bf Git \footnotemark} - ferramenta de controle de versão de projetos;
            \footnotetext{Site oficial: \url{www.git-scm.com}}
            
            \item {\bf \LaTeX \footnotemark} - programa de formatação de texto muito utilizado no
                                               mundo acadêmico; e
            \footnotetext{Site oficial: \url{www.latex-project.org}}
            
            \item {\bf SSH \footnotemark} - protocolo de rede que permite conexões seguras entre
                                            computadores.
            \footnotetext{
                Sobre: \url{pt.wikipedia.org/wiki/SSH};
                Site oficial: \url{http://www.ssh.com}
            }
        
        \end{itemize}
        
        Houve também um processo seletivo de novos membros que envolveu a confecção de quatro
        mini-projetos por parte dos candidatos. Tendo sido orientados pelo membros veteranos do
        grupo durante o processo, esses candidatos puderam adquirir uma experiência breve porém real
        de como é desenvolver um jogo, independentemente de terem sido posteriormente admitidos como
        membros ou não.
        
        Além dessas atividades, alguns integrantes do grupo ainda se organizam para participar de
        competições de desenvolvimento de jogos eletrônicos de tempos em tempos.
    
    \subsection{Metodologia}
        Para gerenciar as atividades do grupo, é seguida uma metodologia de trabalho derivada do
        {\it Scrum}. Um dos membros do grupo fez como Trabalho de Conclusão de Curso no Bacharelado
        em Ciência da Computação justamente um estudo sobre como {\it Scrum} poderia ser aplicado à
        equipes de desenvolvimento de jogos, e usou o USPGameDev como caso de estudo. Portanto, para
        maiores detalhes sobre como a nossa metodologia funciona, recomenda-se a leitura da
        monografia resultante desse trabalho, que pode ser encontrada em:
        
        \begin{center}
            \footnotesize
            \url{http://www.linux.ime.usp.br/~vkdaros/mac499/files/scrum_for_games.pdf}
        \end{center}

\section{\LARGE Histórico}
    % TODO

\section{\LARGE Membros}
    % TODO

\section{\LARGE Projetos}
    % TODO

%%% Parte II: Resultados. %%%

\section{\LARGE Resultados}

% End of document.
\end{document}

