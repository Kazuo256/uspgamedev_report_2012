\documentclass[12pt,onecolumn,a4paper]{article}
\usepackage[brazilian]{babel}
\usepackage[utf8]{inputenc}
\usepackage{hyperref}
%\usepackage{amsmath}


\begin{document}

% Title page.
\begin{titlepage}

    % Title info.
    \title{
        \bf
        \Huge  USPGameDev \\
        \large Grupo de Pesquisa e Desenvolvimento de Jogos da USP \\
        \LARGE Relatório
    }
    \author{Wilson Kazuo Mizutani}
    
    % Print title.
    \maketitle
    
    % No numbering on this page.
    \thispagestyle{empty}
    
\end{titlepage}

% Print table of contents.
\tableofcontents

% Page break.
\clearpage

% New section.
\section{\LARGE O que é o USPGameDev?}

    % Definição e objetivos.
    O USPGameDev é um grupo de pesquisa e desenvolvimento de jogos da USP. Ele tem como meta criar
    um ambiente no qual membros da comunidade USP interessados na criação de jogos eletrônicos
    possam por em prática o que aprenderam em seus cursos e desenvolver suas próprias ideias. Além
    disso, o grupo visa aprender sobre práticas de gerenciamento de equipes e projetos, sobre
    processos de elaboração de jogos e, também, adquirir familiaridade com as atuais ferramentas de
    desenvolvimento utilizadas na área.
    
    % Filosofia.
    O grupo procura adotar uma filosofia de {\it Software Livre} - permitindo que qualquer um tenha
    acesso a estudar, modificar e redistribuir aquilo que produzimos - e de {\it Portabilidade} -
    garantindo que aquilo que produzimos possa ser compatível com a maior parte possível dos
    sistemas operacionais e arquiteturas encontradas nos computadores modernos. É do interesse do
    grupo que o maior número possível de pessoas possam usufruir dos nossos resultados, sem
    restruições.

\subsection{Histórico}

\subsection{Membros}

\subsection{Projetos}

\section{\LARGE Resultados}

% End of document.
\end{document}

